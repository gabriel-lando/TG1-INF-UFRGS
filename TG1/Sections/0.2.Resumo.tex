Com o advento da quinta geração de redes móveis, novas funcionalidades foram adicionadas a essas redes, tornando-as cada vez mais complexas.
O aumento da capacidade dos processadores de propósito geral permitiram que os núcleos de redes móveis pudessem ser executados em software, reduzindo-se os custos de implantação dessas redes. Conteúdo, isso também tornou-as mais suscetíveis a erros de implementação.
Este trabalho realiza uma revisão da literatura trazendo os avanços das redes móveis desde o seu princípio até a quinta geração, fazendo um levantamento sobre testes em núcleos de redes móveis que possuem implementação em código aberto.
Por fim, é proposto um método para testar o desempenho de algumas implementações de núcleos da quinta geração de redes móveis com o intuito de avaliar o comportamento dessas implementações sob alta carga de trabalho.
