Nesta seção serão analisados alguns estudos publicados com relação a testes em redes móveis.
A Tabela \ref{tab:related-works} apresenta um resumo do estudo realizado de trabalhos relacionados apresentados nesta seção.

\subsection{\textit{Tutorial on communication between access networks and the 5G core}}

O estudo publicado por \cite{Dominato2021} cria uma prova de conceito de uma aplicação para testar núcleos \textit{open source} de rede 5G.
Para esse trabalho, foram escolhidos os grupos de testes de conformidade e robustez.
As implementações suportadas por esse testador são \textit{free5GC}, \textit{Open5GS} e \textit{OAI}.

Essa prova de conceito simula uma RAN e um ou mais UEs.
O objetivo do estudo é analisar o comportamento dos diferentes núcleos de rede 5G em dois diferentes cenários. O primeiro cenário foi executando ações de acordo com o que foi especificado pela 3GPP, através de testes de conformidade. O Segundo cenário foi a execução de ações onde informações inválidas foram informadas, forçando o núcleo da rede a contornar essa situação de acordo com o especificado, o que representa os testes de robustez.

Em relação aos testes de conformidade, foram executados 11 diferentes testes, sendo eles teste de registro, autenticação primária e acordo de chave, identificação, transporte, modo de segurança, atualização de configuração genérica do UE, gerenciamento de sessão, gerenciamento de interface, transporte de mensagens NAS, gerenciamento de contexto do UE e gerenciamento de sessão do PDU.
Foi utilizada a \textit{Release 16} do 3GPP como base para a execução dos testes.
O núcleo \textit{Open5GS} foi aprovado em todos os testes de conformidade, enquanto que o \textit{OAI} e o \textit{free5GC} não responderam de acordo com a especificação no teste de atualização de configuração genérica do UE.

Em relação aos testes de robustez, 7 testes foram executados, sendo eles teste de registro, autenticação, segurança, seleção SMF, seleção UPF, validação do fluxo NAS e gerenciamento de interface.
No teste de registro, dois casos foram avaliados, onde o equipamento de usuário enviou uma requisição de registro sem o identificador do equipamento, que é um campo mandatório e o envio da requisição com as informações criptografadas. Nesse teste, apenas o núcleo \textit{Open5GS} respondeu as requisições de acordo com a especificação.
No teste de autenticação, foram realizadas duas operações incorretas, o primeiro teste foi realizado enviando informações inválidas como resposta no fluxo de autenticação, enquanto que o segundo teste foi forçando uma falha de sincronização entre o núcleo e o UE. Nesse teste, todos os núcleos testados responderam de acordo com a especificação da 3GPP.
Para o teste de segurança, foram testadas duas configurações de segurança inválidas, sendo elas responder a uma solicitação de segurança sem enviar um parâmetro obrigatório na resposta e responder a solicitação sem retransmitir as informações de segurança da requisição. Nesse teste, nenhum dos núcleos avaliados respondeu de acordo com a especificação.
Os teste de seleção SMF e seleção UPF representam um equipamento de usuário fazendo uma requisição para estabelecer um fluxo PDU, porém enviando informações mandatórias inválidas. No teste de seleção SMF apenas o \textit{free5GC} se comportou de forma esperada. Todavia, no teste de seleção UPF, todos os núcleos testados se comportaram de acordo com o esperado.
O teste de validação do fluxo NAS visa verificar se o núcleo da rede se comporta da forma esperada caso a requisição de estabelecimento de sessão PDU seja realizada enviando-se as informações necessárias em uma ordem incorreta. Nesse teste, os núcleos \textit{free5GC} e \textit{Open5GS} se comportaram de acordo com a especificação.
Por fim, o teste de gerenciamento de interface simula o registro de uma nova RAN na rede 5G, porém enviando informações inválidas para o AMF, o qual deve rejeitar essa requisição. Apenas os núcleos \textit{free5GC} e \textit{Open5GS} se comportaram de acordo com a especificação.

O estudo considera que todos os núcleos testados obtiveram um desempenho satisfatório para a operação dentro da normalidade. Entretanto, dentre os três núcleos analisados, o \textit{OAI} foi considerado pelos autores do artigo como o núcleo com a implementação menos madura até o momento da publicação.

Os testes realizados no artigo de Dominato et al. são de extrema importância para avaliar o funcionamento de um núcleo de rede 5G.
Como descrito brevemente no artigo, essa implementação de prova de conceito também suporta a realização de um teste onde múltiplos equipamentos de usuário são conectados no núcleo da rede em forma de fila (realizando uma conexão por vez), afim de avaliar se a implementação em teste do núcleo da rede se comporta de acordo com a especificação em casos onde existam diversos UE conectados na rede.
Entretanto, não foram executados testes para comparar o desempenho dessas implementações de núcleo em casos onde uma grande quantidade de UEs se conectem ao núcleo de forma paralela, onde múltiplos equipamentos de usuário façam requisições de registro e autenticação em um curto intervalo de tempo.
Esse teste é importante para avaliar qual implementação possui a melhor performance em casos de alta carga de trabalho.


\textbf{Reproduzir Figura 2 do artigo do Cristiano, explicando os detalhes do core e os protocolos. Usar esses passos para detalhar os componentes (APIs, protocolos...)} \textcolor{red}{Já adicionei ela na seção passada...}

\subsection{\textit{Realizing 5G Network Slicing Provisioning with Open Source Software}}

O artigo de \cite{Lee2021} trata sobre fatiamentos de rede 5G utilizando-se da implementação \textit{open source free5GC} do núcleo de rede 5G.
A ideia de fatiar uma rede permite criar múltiplas redes lógicas sobre uma única rede física e cada rede lógica tendo suas próprias funções de rede e configurações.
Lee et al. executa cinco experimentos, em duas categorias distintas de testes. Primeiramente, o autor executa um experimento de conformidade, verificando se o fatiamento da rede 5G está de ocorrendo conforme o especificado. Após, são executados quatro experimentos medindo o desempenho do núcleo dessa implementação de rede 5G utilizando-se do fatiamento de rede.

O primeiro experimento executado pelo autor tem como objetivo verificar se os pacotes de dados trafegam por todos os nós na ordem correta, conforme a especificação.
Para isso, o autor utiliza de ferramentas de análise de rede, como o \textit{tcpdump}\footnote{\url{https://www.tcpdump.org}}.
O teste foi bem sucedido, permitindo que os testes de desempenho pudessem ser realizados.

O Segundo teste teve como objetivo medir a qualidade da rede utilizando-se de aplicações específicas para sobrecarregar a rede de dados e medir a vazão da rede 5G.
O terceiro experimento visou medir o desempenho das operações de fatiamento de rede simulando um ambiente de banda larga móvel melhorada, onde o UE executa serviços que geram um alto tráfego de dados na rede.
Nesse experimento, foi observado o tráfego de dados em cada dispositivo e fatia da rede.

O quarto experimento simulou uma rede de sensores, onde diversos dispositivos IoT publicaram informações através do protocolo Message Queuing Telemetry Transport (MQTT)\footnote{\url{https://mqtt.org}} e diversas aplicações se inscreveram para receber as mensagens enviadas.
Foram coletadas métricas de latência, perda de pacotes, uso de CPU e tempo de execução de execuções do experimento variando-se a quantidade de dispositivos se publicando e aplicações se inscrevendo.
Constatou-se que a latência aumentava gradativamente a cada aumento na quantidade de conexões, assim como em certo momento houve perda de pacotes devido a sobrecarga da rede.

O quinto experimento consistiu em criar fatias de rede variando-se a quantidade de funções de rede.
O objetivo desse experimento foi simular serviços de diferentes complexidades, medindo-se a latência gerada ao adicionar mais funções de rede.

Os experimentos executador por Lee et al. são muito importantes para analisar como a sobrecarga de redes móveis afeta o desempenho de redes 5G fatiadas.
Entretanto, esse artigo está realizando os experimentos sobre o plano de usuário da rede, não levando em consideração o quanto o plano de controle da rede foi afetado.

\subsection{\textit{Delay metrics and delay characteristics: A study of four Swedish HSDPA+ and LTE networks}}

O estudo de \cite{Garcia2015} realiza testes de desempenho no plano de usuário de redes móveis reais de gerações passadas.
As métricas coletadas pelo estudo foram latência e largura de banda das redes analisadas.
Também foi observado o tempo de carregamento de páginas na internet em diversos cenários, simulando o uso real da rede.

O estudo de Garcia et al. analisa somente o tráfego de dados em redes móveis, verificando o comportamento das redes em uso normal.
Uma vez que as redes testadas foram redes reais de quatro operadoras locais, entende-se as limitações do estudo e sabe-se que a coleta de métricas relacionadas ao plano de controle da rede não seria possível.
Devido a limitações na realização de testes em redes reais, o autor deste estudo não pôde realizar testes de desempenho durante o processo de ingresso na rede móvel, com intuito de testar o plano de controle.



\begin{table}[]
\centering
\caption{Tabela comparando os trabalhos relacionados}
\label{tab:related-works}
\begin{tabular}{llll}
\hline
\textbf{Autor} & \textbf{Tipos de teste} & \textbf{\begin{tabular}[c]{@{}l@{}}Implementações\\ testadas\end{tabular}} & \textbf{Protocolos testados} \\ \hline
\cite{Dominato2021} & \begin{tabular}[c]{@{}l@{}}Conformidade\\ e robustez\end{tabular} & \textit{\begin{tabular}[c]{@{}l@{}}free5GC, Open5GS\\ e OAI\end{tabular}} & NAS e NGAP \\ \hline
\cite{Lee2021} & \begin{tabular}[c]{@{}l@{}}Conformidade\\ e desempenho\end{tabular} & \textit{free5GC} & Fluxo PDU \\ \hline
\cite{Garcia2015} & Desempenho & Não se aplica & Fluxo PDU \\ \hline
\end{tabular}
\end{table}
